% \textbf{\textit{What processes control the generation of biodiversity?}}
% My research group is motivated to
\noindent
The central themes motivating my research group are
(1) better understanding the processes that generate the amazing biological
diversity of our planet,
and
% (2) incorporating computational exercises into biology curricula to build 
% intuition for evolutionary processes and better train the next generation of
% computational biologists, and
(2) integrating computational research into biology
curricula\footnote{\label{Wright19}\shortfullcite{Wright2019}\hspace{-0.8em}}.
% especially for students from historically excluded groups.
% \textbf{\textit{What processes are responsible for the generation and
%         distribution of biodiversity?}}
% Elucidating answers to this question, in its broadest sense, is the primary
% motivation of my research group.
While research interests in my lab are diverse, our work is unified by
using
% our
% passion for organismal biology and
% use of
phylogenetics to infer and explicitly account for shared ancestry in our
approaches to better understand processes of evolution.
We integrate many methods in this endeavor, including
% collection-based
field work,
genomics,
developing statistical methods and software for inferring
evolutionary history, and ultimately applying these computational tools to
genomic data to test
hypotheses about processes that generate biodiversity.
The questions my lab explore are integrative, open-ended, and require critical
thinking and creativity.
% As a result,
I strive to include students and collaborators with diverse backgrounds and
perspectives to bring together unique insights.

\section*{Previous Research}
\subsubsection*{Genomic approaches to testing climate-driven diversification}
A question I have been passionate about since graduate school is whether
Quaternary climatic oscillations promoted diversification by fragmenting the
distributions of species.
An ideal system for addressing this question is the dynamic landscape of
Southeast Asia, where many groups of islands coalesced during lower sea levels
of glacial periods and were fragmented during interglacial rises in sea level.
It has often been assumed that this repeated fragmentation functioned as a
``species pump'' driving vicariant speciation across these islands.
If correct, this predicts that population divergences across islands fragmented
by rising sea levels were temporally clustered around interglacial sea
incursions.

To test this prediction, we
(1) developed
simulation\footnote{\label{Oaks14dpp}\shortfullcite{Oaks2014dpp}\hspace{-0.8em}}
and
likelihood-based\footnote{\label{Oaks18ecoevolity}\shortfullcite{Oaks2018ecoevolity}\hspace{-0.8em}}
statistical methods for inferring shared evolutionary events from comparative
genomic data, and
(2) applied our methods to genomic data from geckos we sampled from 16 pairs of
islands in the Philippines, many of which were connected during glacial
periods.
We found strong support that the pairs of gecko populations diverged
independently,
refuting the predictions of the climate-driven species-pump
model\footnote{\label{Oaks18paic}\shortfullcite{Oaks2018paic}\hspace{-0.8em}};
a result that makes sense in light of the natural history and habitat
preferences of these lizards.

The entire arc of this line of inquiry was funded by the National Science
Foundation (NSF)
(DEB
\href{https://www.nsf.gov/awardsearch/showAward?AWD_ID=1011423}{1011423},
\href{https://www.nsf.gov/awardsearch/showAward?AWD_ID=1308885}{1308885},
and
\href{https://www.nsf.gov/awardsearch/showAward?AWD_ID=1656004}{1656004})
and is a good example of my lab's willingness to derive and implement new
models and computational approaches to interrogate questions that motivate us.
Our full-likelihood model that efficiently uses information
from genome-scale data was a step forward for biogeography.
However,
it was limited in two ways.
First,
by focusing on pairs of populations, it ignores the possibility of patterns of
co-divergences across the ancestral history shared among the populations.
To overcome this limitation, my research group is currently working on fully
phylogenetic approaches to testing for temporally clustered
divergences\footnote{\label{Oaks21phycoeval}\shortfullcite{Oaks2021phycoeval}\hspace{-0.8em}}.
Second, it assumes the natural history of organisms does not affect how they
diversify, which is why my group is developing process-based evolutionary
models that allow diversification dynamics to vary among lineages depending on
their characteristics.
% I also documented our progress in all aspects of this work in open-science
% project repositories that can be found at
% \href{https://github.com/joaks1/msbayes-experiments}{github.com/joaks1/msbayes-experiments},
% \href{https://github.com/phyletica/ecoevolity-experiments}{github.com/phyletica/ecoevolity-experiments},
% and
% \href{https://github.com/phyletica/gekgo}{github.com/phyletica/gekgo}.

%   Current work
    %   Generalizing Bayesian phylogenetics (NSF-funded)
        %   We are making assumption in phylogenetics
        %   Working to relax this assumption and explore full space of phylogenetic
        %   trees
        %   This will develop a statistical foundation for testing predictions of
        %   many interesting biological processes
        %   Genome evolution, epidemiology, biogeography
        %   Applying to gekkonid genomic data to test hypotheses about biogeography

    %   Generalizing the birth-death process

    %   Coevolution effect
        %   Resubmitting to NSF EEID this Fall

    %   Continuing work on shared divergences

    %   Conservation genetics

\section*{Current and Future Research}
% Below, I summarize some of the on-going and developing projects in my lab,
% including plans for funding my group's research.
% Given my current position, these funding plans are geared toward sources in the
% United States.
% However, if I were fortunate enough to join the
% I am confident that the research summarized below would
% be competitive for
% funding opportunities in


\subsubsection*{Generalizing phylogenetics to infer shared evolutionary events}
In phylogenetics, we have long made a limiting assumption about the processes
that cause evolutionary lineages (whether they are genes or populations) to
diversify: \textit{we assume divergences across the tree of life are independent}.
Violations of this assumption are likely common and span many areas of biology,
including biogeography, gene-family evolution, epidemiology, and symbiont
coevolution.
Prior to our work, we lacked a statistical framework for testing the
predictions of the diverse and interesting processes of diversification that
violate this assumption.
With funding from the NSF
(\href{https://www.nsf.gov/awardsearch/showAward?AWD_ID=1656004&HistoricalAwards=false}{DEB 1656004})
we created such a framework by generalizing the space of phylogenies that is
explored during Bayesian phylogenetic inference,
allowing us to approximate the probability
of patterns predicted by processes that cause non-independent divergences 
across the tree of
life\cref{Oaks21phycoeval}.
% By exploring the full space of phylogenies, we can approximate the probability
% of patterns predicted by processes that cause nonindependent divergences 
% across the tree of life.
% We have implemented this approach in 
% \href{http://phyletica.org/ecoevolity/}{ecoevolity}
% and are currently testing it using simulations.

When we applied our generalized phylogenetic approach to comparative genomic data
from two genera of geckos from across the Philippine Islands, we found support
for shared and multifurcating divergences over the past 4 million years,
as predicted by the repeated fragmentation of the islands by interglacial rises
in sea
level\cref{Oaks21phycoeval}.
Our findings confirmed the importance of looking beyond pairs of populations to
their full ancestral history when testing for non-independent divergences.
Our fully phylogenetic approach also opens the door to testing rates and
patterns of divergence predicted by processes of interest across many fields of
biology,
enabling my lab group to explore new research directions in methods
development, biogeography, and epidemiology.

% One motivation for developing this new statistical phylogenetic framework, is
% our desire to test several long-standing biogeographical hypotheses to explain
% the distribution of the rich biodiversity across Southeast Asia.
% We are currently collecting genomic data from three genera of geckos that are
% co-distributed across much of this region.
% Our recent fieldwork in Indochina has added to our geographic sampling of these
% three genera, and will allow us to use our new methods to test for patterns
% predicted by processes of diversification across both mainland and insular
% Southeaset Asia.
% To make our approach more generally applicable to questions about biogeography,
% genome evolution, epidemiology, and coevolution,
% we are also porting our distributions and algorithms to the flexible
% statistical phylogenetic software package
% \href{https://revbayes.github.io/}{RevBayes}.

\subsubsection*{A generalized, trait-based, birth-death process}
To create a process-inspired phylogenetic framework to infer rates and patterns
of co-diversification,
my lab group is collaborating with Sebastian H\"ohna to develop a
% My lab group is currently developing a
generalized, state-dependent birth-death (BD) process, which we call the
birth-death-burst (BDB) process.
To relax the assumption of independent bifurcating divergences, we extended the
BD process to include a variable number of ``bursts'' of shared divergences through time that
occur at rate $\burstrate$, while other divergences occur
at rate $\speciationrate$.
% This process-based approach enables us to learn about biologically
This enables researchers to learn about biologically
meaningful macroevolutionary parameters that govern processes of
diversification.
For example, to test whether macroevolutionary processes effect
lineages differently based on their characteristics, the BDB model will allow
different \speciationrate and \burstrate parameters depending on the traits of
the evolving lineages. 
% To determine the set of trait-based models that best explain the data, we will
% use Bayesian model averaging via reversible-jump Markov chain Monte Carlo.
% Additional funding for this work is included in my pending NSF CAREER
% proposal.
To support this work, I have a pending proposal submitted to the NSF DEB
Systematics and Biodiversity Science Cluster.

\subsubsection*{Estimating the impact of super-spreading events on \covid}
Understanding the role of social gatherings in spreading infectious disease
has important health implications for both humans and wildlife.
Multiple transmissions of a pathogen at a large gathering leave distinct
patterns of shared divergences in the phylogenetic history of the pathogen.
Using viral sequences and the corresponding times samples were collected, our
BDB method will enable us to estimate the rate of transmissions at social
events (\ie, shared divergences; \burstrate) relative to the rate of individual
transmissions (\speciationrate),
providing a proxy for the relative importance of social gatherings
in the spread of a pathogen.

To jointly infer the \sarscov phylogeny, shared divergences,
and macroevolutionary parameters,
we will analyze regional sequence and sample-time data with the
state-dependent BDB model we are developing in the software package \revbayes.
From the posterior samples of regional analyses, we will summarize the rate of
shared divergences across windows of time to quantify the effect of holidays
on the spread of the virus at social gatherings.
To test whether there are differences in the rate of spread via social
gatherings among the named variants of \sarscov,
we will use Bayes factors to compare the probability of strain-dependent to
strain-independent BDB models.
To validate the applicability of the BDB model to epidemiology,
we will determine whether it can identify shared divergences when applied to
\sarscov datasets from epidemiologically validated super-spreading events.
% I am seeking funding for this work from the NSF as part of my pending CAREER
% proposal, and am working on a proposal for an R21 award from the NIH.
To fund this work, I am working on a proposal for an R21 award from the NIH.
% Investigator Initiated Research in Computational Genomics and Data Science


\subsubsection*{Diversification of geckos across Southeast Asia}
One motivation for developing the BDB model is to improve our understanding of
how large-scale changes in geography affect diversification, a primary goal of
biogeography.
A major gap in our understanding of biogeography is how these processes
interact with the ecology of organisms in generating patterns of biological
diversity.
To begin filling this gap,
my research group will
apply our BDB model to genomic data from
hundreds of bent-toed geckos (\textit{Cyrtodactylus}) from
across Southeast Asia
to test whether
biogeographical processes
have affected lineages differently, depending on their habitat preferences.

Southeast Asia has a dynamic geological history, including uplift and
subsequent erosion of marine sediments from the tectonic collision of the
Indian Subcontinent.
% over the last 25 million years.
% \cite{Rohling1998,Voris2000,Woodruff2003,deBruyn2004,Naish2009,Woodruff2010}.
Comprising
% $\approx380$
more than 380
species that span nearly all of South and Southeast
Asia,
bent-toed geckos (\textit{Cyrtodactylus}) are an ideal system for studying
the effects of this region's dynamic landscape on the generation of
biodiversity.
\textit{Cyrtodactylus} are ecologically diverse, ranging from generalists
to species highly adapted to at least 10 microhabitats.
% \cite{Grismer2021diversity}.
Karst specificity evolved at least 24 times and currently
accounts for 25\% of the species in the genus,
% \cite{Grismer2021diversity},
suggesting a higher rate of diversification in karst-specific
lineages, especially considering that limestone karst comprises a tiny fraction
of the landscape of South and Southeast Asia.
Compared to other habitat preferences,
karst-specific \textit{Cyrtodactylus} species show remarkable levels
of microendemism, with many species being restricted to a single
karst formation.
% \cite{Grismer2021diversity,Grismer2020ee,Grismer2020salweencyrts,Grismer2018}.
We hypothesize the high levels of diversity and endemism of karst-specific
species was caused by the rapid fragmentation of limestone karst
habitat following its uplift.
% \cite{Gillieson2005,Price2014,Sholihah2021}.
% Examples of such fragmentations include the establishment and changes in course
% of major river systems
% (\eg, Ayeyarwady, Chiang Mai, Mekong, Red, and Salween)
% that carved through and isolated limestone karst formations.
% Such large-scale processes would have affected multiple gecko lineages, causing
% shared divergences.
If this explanation is correct, karst-adapted lineages should have higher rates
of shared divergences (\burstrate) compared to geckos living in other habitats.
% Funding for this work is included in my pending NSF CAREER proposal.
To support this work, I have a proposal pending at the NSF DEB.
% My recent proposal to NSF DEB to support this work was not funded, but was
% well-reviewed; I am currently preparing it for resubmission.

% \subsubsection*{Integrating simulation-based research into prison STEM education}
%             % Integrate coding-to-learn exercises into the evolution
%             % course I teach at Alabama prisons that will culminate with the
%             % students designing simulations to ``stress test'' the BDB method.

% % Adult prisoners are among the most underserved populations in this
% % country,
% % but evidence suggests that society as a whole has much to gain from prisoners
% % receiving a quality education \cite{Vacca2004}.
% My lab group is strongly committed to providing educational opportunities to
% the most underserved population in this country, incarcerated people.
% Since 2017, members of my research group and I have taught evolutionary and
% organismal biology to students incarcerated in Alabama correctional facilities
% (please see teaching statement).
% These facilities restrict the technology we can take into the classroom, which
% has prevented us from using computers to teach our students.
% To get around this challenge, we have obtained approval to allow students to use
% single-board computers in prison classrooms.
% % These are small, inexpensive, and can be purchased without
% % a housing and wireless connectivity,
% % which is key to getting facilities to approve their use, because they cannot be
% % used to hide contraband or connect to networks.
% This will enable us to teach the students computer skills and use
% coding-to-learn exercises to help them gain intuition for how processes of
% evolution work and interact.
% A major learning objective of these computational exercises is to understand
% how microevolutionary processes give rise to macroevolutionary patterns of
% biological diversity.

% The coding exercises will culminate in a capstone activity
% toward the end of the 14-week course,
% when the students design evolutionary scenarios and write code to simulate
% genetic data under them.
% Their
% % forward-time
% simulations will begin with a single ancestral species
% that gives rise to a clade of descendants.
% % only some of which will survive.
% At this stage in the course, the students will be familiar with
% micro and macroevolutionary processes and exploring their behavior
% with simulations.
% The emphasis will be on fun and creativity in designing their unique scenario
% or sets of scenarios.
% Outside of class, we will apply the BDB model
% to the datasets generated under the students' models.
% During the last week of class, we will bring the results to show them how well
% our approach did at inferring the evolutionary histories they created.

% The evolutionary scenarios the students design
% % using a forward-time, population-genetics simulator
% % like \slim
% will be far removed from the assumptions
% of the macroevolutionary BDB model,
% making the simulated data a good proxy for real biological data.
% % As a result, the data generated under the students' models will be a good proxy
% % for real, biological data.
% If the BDB model performs well on these data, we, and other scientists, can be
% more confident in the inferences it makes from real data.
% % This level of ``stress-testing'' methods is rarely done in biology, and will
% % provide an important assessment of our new methods' behavior.
% Including our students in
% cutting-edge research in computational biology will not only help
% teach them about evolutionary biology, but also the process of science.
% All of our students will be co-authors on the papers that present the results
% of these simulation-based analyses.
% I am currently seeking funding for this work as part of my pending NSF CAREER
% proposal.
% I am also working on similar approaches to incorporating computational
% biology research into undergraduate and graduate courses, and am
% confident these activities can attract funding from programs at the NSF and NIH
% that emphasize education and training.

\subsubsection*{Integrating computational research into biology classrooms}
            % Integrate coding-to-learn exercises into the evolution
            % course I teach at Alabama prisons that will culminate with the
            % students designing simulations to ``stress test'' the BDB method.

% Adult prisoners are among the most underserved populations in this
% country,
% but evidence suggests that society as a whole has much to gain from prisoners
% receiving a quality education \cite{Vacca2004}.
My lab group is strongly committed to integrating computational biology
research into undergraduate evolution courses.
Our goal is to teach students basic computational skills and use
coding-to-learn exercises to help them gain intuition for how processes of
evolution work and interact.
A major learning objective of these computational exercises is to understand
how microevolutionary processes give rise to macroevolutionary patterns of
biological diversity.
The coding exercises we are designing will culminate with a capstone activity
toward the end of the course, for which the students will design evolutionary
scenarios and write code to simulate genetic data under them.
Their
% forward-time
simulations will begin with a single ancestral species
that gives rise to a clade of descendants.
% only some of which will survive.
At this stage in the course, the students will be familiar with
micro and macroevolutionary processes and exploring their behavior
with simulations.
The emphasis will be on fun and creativity in designing their unique scenario
or sets of scenarios.
Outside of class, we will apply the BDB model
to the datasets generated under the students' models.
During the last week of class, we will bring the results to show them how well
our approach did at inferring the evolutionary histories they created.

The evolutionary scenarios the students design
using a forward-time, population-genetics simulator
% like \slim
will be far removed from the assumptions
of the macroevolutionary BDB model,
making the simulated data a good proxy for real biological data.
% As a result, the data generated under the students' models will be a good proxy
% for real, biological data.
If the BDB model performs well on these data, we, and other scientists, can be
more confident in the inferences it makes from real data.
% This level of ``stress-testing'' methods is rarely done in biology, and will
% provide an important assessment of our new methods' behavior.
Including our students in
cutting-edge research in computational biology will not only help
teach them about evolutionary biology, but also the process of science.
All of our students will be co-authors on the papers that present the results
of these simulation-based analyses.
% I am currently seeking funding for this work as part of my pending NSF CAREER
% proposal.
I am currently seeking funding for this work as the broader impact component of
my NSF proposals.
% I am also working on similar approaches to incorporating computational biology
% research into other undergraduate and graduate courses, and am confident these
% activities can attract funding from programs at the NSF and NIH that emphasize
% education and training.



% \subsubsection*{Bringing phylogenetics into the inference of population structure}
% Gideon Bradburd and colleagues recently introduced a method that uses genomic
% data to assign individuals to populations while simultaneously accounting for
% continuous patterns of genetic variation across a
% landscape\footnote{\label{Bradburd18}\shortfullcite{Bradburd2018}\hspace{-0.8em}}.
% This was an important step forward from methods that ignored continuous spatial
% genetic differentiation.
% However, their model assumes that the allele frequencies in each population
% have drifted away from an ancestral population independently of one another
% (i.e., ignoring shared ancestry among the populations).
% Gideon and I are working on extending the method implemented in conStruct
% to simultaneously infer
% (1) the population assignment of each individual's genome (allowing their
% genomes to be admixed),
% (2) spatial genetic differentiation within each population, and
% (3) the phylogenetic relationships of the populations.
% We are planning to submit a proposal to the NSF DEB
% % Systematics and Biodiversity Science Cluster
% Evolutionary Processes Cluster
% to fund this work.


    %   Continued refinement of testing for shared divergences
        %   More nonparametric models
        %   Analytical integration of population sizes
        %   Modeling linked loci
        %   Gene flow
        %   Current approach ignores gene flow and assumes SNPs
        %   Working with Yujin to get at this

    %   Functional phylogeography
        %   Phylogeography with whole genomes creates opportunities to infer
        %   spatial patterns of selection
